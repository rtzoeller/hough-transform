\documentclass[letterpaper,12pt,titlepage]{article}

\title{Line detection via the Hough transform}
\date{December 23, 2015}
\author{Ryan Zoeller}

\begin{document}
\maketitle
\newpage

\section{Abstract}
The classical Hough transform provides a convenient technique for detecting lines in an image.
A thresholded edge matrix is generated for the image (using any one of a number of edge detection
operators, e.g. the Sobel operator); this matrix is then used in a voting process carried
out in a parameter space. In the classical Hough transform this parameter space is the polar
normal form of a line, which is especially useful as it allows for the constrained
search space $\theta \in (-\pi,\pi]$ as well as for representation of vertical lines.
This paper presents a multithreaded implementation of the classical Hough transform, which assigns
each thread a sub-interval of the search space. Due to the partitioning of the vote accumulation
matrix, each thread can operate on this sub-interval without synchronization, resulting in a dramatic
performance increase on dense inputs.

\end{document}
